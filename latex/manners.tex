\section{Le cinque buone maniere}

\begin{enumerate}

%1
\item \textbf{Rispetta sempre gli standard} \\
È sempre buona norma utilizzare standard per l’impostazione del codice. I \emph{design pattern} della Gang of Four sono un ottimo esempio. Potresti guadagnarti la stima e il rispetto dei colleghi, se produci software di qualità; viceversa, se ti lanci verso un \emph{hackathon} di \emph{spaghetti-code}, probabilmente finirai per essere il bersaglio durante il gioco a freccette nella sala relax aziendale.

%2
\item \textbf{Proteggi il \emph{branch} principale e segui un \emph{versioning} ben preciso} \\
Non lasciare che il \emph{branch} \texttt{master} sia accessibile da tutti: solo una persona (con una \emph{seniority} adeguata) deve avere l’onere di eseguire l’allineamento del \emph{branch} principale. Una volta scelto il \emph{pattern} di \emph{versioning}, seguilo sempre: evita eccezioni o di uscire dai binari. Prima di allineare i \emph{branch} condivisi (per esempio \texttt{develop}, \texttt{uat}, \texttt{collaudo}), testa il codice in maniera meticolosa.

%3
\item \textbf{Commenta e indenta il codice adeguatamente} \\
Ad inizio progetto, bisogna allineare le configurazioni di indentazione di ogni membro del team. Di solito gli \textsc{ide} ti consentono di sovrascrivere le configurazioni di \emph{default}. Sceglietene una e allineate tutti gli \textsc{ide} dei componenti del team. È anche un bel gesto commentare (senza scrivere un poema) il codice sorgente: i colleghi apprezzeranno. Creare una documentazione del codice sorgente, soprattutto per le componenti riusabili e generiche, è altrettanto un gesto apprezzato. Se segui questo consiglio, hai l’opportunità di diventare la \emph{rock star} del tuo \emph{open space}. Non mi farei sfuggire questa occasione.

%4
\item \textbf{Coinvolgi subito i colleghi interessati se sospetti un \emph{merge-conflict}} \\
Se stai sviluppando codice che impatta file sorgente attualmente in lavorazione (o già implementati) di altri componenti del team, coinvolgi subito i colleghi per evitare conflitti successivamente. Oltre ad essere una buona maniera, potresti evitare di perdere una numerosa quantità di codice già sviluppato.

%5
\item \textbf{Proponi \emph{pair-programming} in caso di \emph{refactoring}}

Se trovi del codice da refattorizzare e vuoi continuare il tuo percorso verso la santità, refattorizzalo. Sarebbe una grande azione anche coinvolgere l’autore del codice da refattorizzare: un buon sano \emph{pair-programming} migliorerà le \emph{skill} di programmazione di entrambi.

\end{enumerate}